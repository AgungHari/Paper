% Mengubah keterangan `Abstract` ke bahasa indonesia.
% Hapus bagian ini untuk mengembalikan ke format awal.
% \renewcommand\abstractname{Abstrak}

\begin{abstract}

  % Ubah paragraf berikut sesuai dengan abstrak dari penelitian.
  The development of autonomous wheelchairs has become increasingly important in improving mobility for individuals with limited mobility. This study proposes the development of a YOLOv8-based autonomous wheelchair system for obstacle avoidance, specifically focusing on human obstacle detection. By utilizing the advanced object detection capabilities of YOLOv8, the proposed system aims to effectively detect and avoid human obstacles. The system detects humans through video using an Intel NUC and a camera. When an obstacle is detected, the NUC sends a command to the ESP32 to operate the motor to perform avoidance maneuvers. Performance testing of the avoidance success was conducted with 30 trials on stationary human objects. The test results showed that the wheelchair successfully avoided obstacles 30 times, providing a success rate of 100\%. This indicates that the designed autonomous wheelchair system is capable of performing obstacle avoidance without mistake.
  

\end{abstract}

% Mengubah keterangan `Index terms` ke bahasa indonesia.
% Hapus bagian ini untuk mengembalikan ke format awal.
% \renewcommand\IEEEkeywordsname{Kata kunci}

\begin{IEEEkeywords}

  % Ubah kata-kata berikut sesuai dengan kata kunci dari penelitian.
  Autonomous Wheelchair, YOLOv8, Intel NUC, ESP32, Human Detection, Mobility Aid.

\end{IEEEkeywords}
