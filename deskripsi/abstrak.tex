% Mengubah keterangan `Abstract` ke bahasa indonesia.
% Hapus bagian ini untuk mengembalikan ke format awal.
\renewcommand\abstractname{Abstrak}

\begin{abstract}

  % Ubah paragraf berikut sesuai dengan abstrak dari penelitian.
  Pengembangan kursi roda otonom telah menjadi semakin penting dalam meningkatkan mobilitas bagi individu dengan mobilitas terbatas. Studi ini mengusulkan pengembangan sistem kursi roda otonom berbasis YOLOv8 untuk menghindari obstacle, khususnya fokus pada deteksi obstacle manusia. Dengan memanfaatkan kemampuan deteksi objek yang canggih dari YOLOv8, sistem yang diusulkan bertujuan untuk mendeteksi dan menghindari obstacle manusia secara efektif. Sistem tersebut mendeteksi manusia melalui video menggunakan Intel NUC dan Kamera. Obstacle yang terdeteksi membuat NUC mengirim perintah ke ESP32 untuk menjalankan motor untuk melakukan manuver penghindaran. Pengujian performa keberhasilan penghindaran dilakukan dengan 30 kali percobaan pada objek manusia yang diam. Hasil pengujian menunjukkan bahwa kursi roda berhasil menghindar sebanyak 30 kali tanpa gagal, memberikan tingkat keberhasilan sebesar 100\%. Hal ini menunjukkan bahwa sistem kursi roda otonom yang dirancang mampu melakukan penghindaran rintangan dengan sangat baik.

\end{abstract}

% Mengubah keterangan `Index terms` ke bahasa indonesia.
% Hapus bagian ini untuk mengembalikan ke format awal.
\renewcommand\IEEEkeywordsname{Kata kunci}

\begin{IEEEkeywords}

  % Ubah kata-kata berikut sesuai dengan kata kunci dari penelitian.
  Kursi Roda Otonom, YOLOv8, Intel NUC, ESP32, Deteksi Manusia, Bantuan Mobilitas

\end{IEEEkeywords}
