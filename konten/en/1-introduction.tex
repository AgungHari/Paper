    % Ubah judul dan label berikut sesuai dengan yang diinginkan.
    \section{Introduction}
    \label{sec:introduction}

    % Ubah paragraf-paragraf pada bagian ini sesuai dengan yang diinginkan.

    Mobility is a crucial aspect of daily life that enables individuals to perform various activities independently. However, for individuals who experience paralysis or other mobility impairments, reliance on assistive devices such as wheelchairs becomes inevitable. According to the Indonesian Dictionary (KBBI), paralysis is a condition characterized by the loss of movement function in certain parts of the body, which can be caused by various factors, such as injuries or nerve diseases \cite{Daring_2016}. Medically, paralysis can occur due to damage to the nervous system, whether central or peripheral, resulting in the loss of motor control in body parts \cite{Pansawira_2022}.

    Technological advancements in this field have significantly contributed to improving the quality of life for people with disabilities. One notable innovation is the development of electric wheelchairs controlled via joystick. Research by Choi, Chung, and Oh demonstrates that integrating motion control of electric wheelchairs with a joystick can enhance user safety and comfort \cite{choi2019motion}. Nevertheless, challenges remain in terms of navigation and obstacle avoidance, especially in dynamic and complex environments.

    With the progress of artificial intelligence technology, deep learning has become a highly effective tool for real-time object detection, identification, and tracking. Lecrosnier's research reveals that employing deep learning in object detection and localization on smart wheelchairs can enhance user mobility and safety in healthcare environments \cite{lecrosnier2021deep}. This technology enables wheelchairs to automatically avoid obstacles, including humans, allowing users to move more safely and efficiently.

    The development of autonomous wheelchairs is a vital step towards improving the independence and quality of life for individuals with mobility limitations. With advancements in sensor technology and image processing, the application of object detection methods such as YOLO (You Only Look Once) has become central to innovative autonomous mobility solutions.

    In this context, this project aims to enhance the navigation of autonomous wheelchairs by integrating YOLOV8 for obstacle detection and avoidance, ensuring user safety and comfort in various situations.