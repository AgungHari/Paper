% Change the following title and label as desired.
\section{Conclusion}
\label{sec:conclusion}

% Modify the paragraphs in this section as desired.

Based on the testing results, the following conclusions can be drawn:

\begin{enumerate}
  \item The model with the highest metrics trained with various configurations is the one with the highest mAP score at IoU 0.5 of 81.85\%. This value is sufficient for performing avoidance, as seen from the excellent avoidance performance results.
  \item The performance of the NUC in FPS testing produced a lower value compared to the author's personal laptop, with a difference of 7.029 fps.
  \item The average delay obtained in the testing was approximately 0.2494 seconds, and the average inference value obtained was 139.4899 ms or 0.1394 seconds.
  \item The results show that detection using \emph{Bounding Box} and shoulder landmarks is more accurate at longer distances (150 cm and 100 cm), while arm landmarks are more accurate at closer distances (50 cm). The best average \emph{difference} for the bounding box at 150 cm is 3.2 cm, the best average \emph{difference} for shoulder landmarks at 100 cm is 2.2 cm, and the best average \emph{difference} for arm landmarks at 50 cm is 1.93 cm.
  \item Detection performance results were satisfactory in 30 test samples, with a success rate of 100\%, indicating that the system created can avoid humans very well.

\end{enumerate}

\section{Suggestions}
\label{chap:suggestions}

For further development in future research, the following suggestions can be given:

\begin{enumerate}

  \item Increase the variety of datasets to enhance detection performance.
  \item Use a device with better performance for higher fps.
  \item Improve the detection grid performance by making more detailed adjustments for better mapping.
  \item Use a device cooler when testing in an open room to avoid performance degradation.
\end{enumerate}
