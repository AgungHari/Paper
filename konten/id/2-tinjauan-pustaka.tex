% Ubah judul dan label berikut sesuai dengan yang diinginkan.
\section{Tinjauan Pustaka}
\label{sec:tinjauanpustaka}

\subsection{Object Detection}
Deteksi objek secara real-time telah muncul sebagai komponen kritis dalam berbagai aplikasi, meliputi berbagai bidang seperti kendaraan otonom, robotika, pengawasan video, dan realitas tertambah. Di antara berbagai algoritma deteksi objek, kerangka kerja YOLO (\emph{You Only Look Once}) telah menonjol karena keseimbangan kecepatan dan akurasi yang luar biasa, memungkinkan identifikasi objek yang cepat dan dapat diandalkan dalam gambar. Sejak diperkenalkan, keluarga YOLO telah berkembang melalui beberapa iterasi, setiap versi membangun atas versi sebelumnya untuk mengatasi keterbatasan dan meningkatkan kinerja.

\subsection{YOLO (\emph{You Only Look Once})}
YOLO oleh \cite{YOLO}. Joseph Redmoon untuk pertama kalinya memperkenalkan pendekatan \emph{End to end} pada deteksi objek secara \emph{Real Time}. Nama YOLO, yang merupakan singkatan dari \emph{"You Only Look Once"} (Anda Hanya Melihat Sekali), merujuk pada kemampuannya untuk menyelesaikan tugas deteksi dengan hanya satu kali laluan jaringan, berbeda dengan pendekatan sebelumnya yang menggunakan jendela geser diikuti oleh pengklasifikasi yang harus dijalankan ratusan atau ribuan kali per gambar atau metode yang lebih canggih yang membagi tugas menjadi dua langkah, di mana langkah pertama mendeteksi kemungkinan daerah dengan objek atau proposal daerah dan langkah kedua menjalankan pengklasifikasi pada proposal tersebut. Selain itu, YOLO menggunakan keluaran yang lebih sederhana berdasarkan hanya regresi untuk memprediksi keluaran deteksi sebagai lawan dari Fast R-CNN yang menggunakan dua keluaran terpisah, sebuah klasifikasi untuk probabilitas dan regresi untuk \emph{box} koordinat\cite{YOLO}

\subsection{YOLOv8}
YOLOv8 diluncurkan pada Januari 2023 oleh \emph{Ultralytics}, perusahaan yang mengembangkan YOLOv5. YOLOv8 menyediakan lima versi skala: YOLOv8n (nano), YOLOv8s (kecil), YOLOv8m (sedang), YOLOv8l (besar), dan YOLOv8x (ekstra besar). YOLOv8 mendukung berbagai tugas visi seperti deteksi objek, segmentasi, estimasi pose, pelacakan, dan klasifikasi\cite{Yolov8}. Arsitektur YoloV8 terdiri dari lapisan Convolution 2D, C2f(Cross Stage Partial Network), SPPF, Upsampling dan Concate. Gambar \ref{fig:Arsitektur Yolov8} merupakan detail arsitektur YOLOv8.

% Contoh input gambar
\begin{figure}[H]
  \centering

  % Ubah dengan nama file gambar dan ukuran yang akan digunakan
  \includegraphics[scale=0.5]{gambar/YoloV8Architecture.jpg}

  % Ubah dengan keterangan gambar yang diinginkan
  \caption{Arsitektur YOLOv8.}
  \label{fig:Arsitektur Yolov8}
\end{figure}


\subsection{Estimasi Pose}
Estimasi pose adalah tugas menggunakan model pembelajaran mesin (\emph{ML}) untuk memperkirakan pose seseorang dari gambar atau video dengan mengestimasi lokasi spasial sendi tubuh utama (titik kunci atau \emph{keypoints}). Estimasi pose merujuk pada teknik visi komputer yang mendeteksi sosok manusia dalam gambar dan video, sehingga seseorang dapat menentukan, misalnya, di mana siku seseorang muncul dalam gambar. Penting untuk menyadari bahwa estimasi pose hanya memperkirakan di mana sendi tubuh kunci berada dan tidak mengenali siapa yang ada dalam gambar atau video.\cite{tensorflow2015-whitepaper}

\subsection{MediaPipe}
MediaPipe adalah sebuah kerangka kerja yang dirancang oleh Google untuk membangun pipeline persepsi secara real-time. MediaPipe memungkinkan pengembang untuk mengintegrasikan berbagai jenis data sensor seperti video, dan data lainnya dalam satu platform yang efisien dan dapat dijalankan di berbagai perangkat, mulai dari mobile hingga desktop \cite{MediaPipe}

Kerangka kerja ini menggunakan konsep "graph" di mana setiap node di dalam graph merupakan sebuah "calculator" yang melakukan tugas spesifik, seperti deteksi objek, pelacakan pose, atau segmentasi gambar. Setiap node ini dapat dikonfigurasi melalui GraphConfig, yang mendeskripsikan topologi serta fungsionalitas dari node-node tersebut.

\subsection{MediaPipe Pose}
MediaPipe Pose (MPP), sebuah kerangka kerja lintas platform sumber terbuka yang disediakan oleh Google, digunakan untuk mendapatkan perkiraan koordinat sendi manusia 2D dalam setiap bingkai gambar. MediaPipe Pose membangun pipa dan memproses data kognitif dalam bentuk video menggunakan pembelajaran mesin (\emph{machine learning} - ML). MPP menggunakan BlazePose yang mengekstrak 33 landmark 2D pada tubuh manusia seperti yang ditunjukkan pada Gambar \ref{fig:Pose MediaPipe}. BlazePose adalah arsitektur pembelajaran mesin ringan yang mencapai kinerja real-time pada telepon seluler dan PC dengan inferensi CPU. Ketika menggunakan koordinat yang dinormalisasi untuk estimasi pose, rasio invers harus dikalikan dengan nilai piksel sumbu-y. \cite{MediapipePose}

% Contoh input gambar
\begin{figure}[H]
  \centering

  % Ubah dengan nama file gambar dan ukuran yang akan digunakan
  \includegraphics[scale=0.5]{gambar/mp_pose.jpg}

  % Ubah dengan keterangan gambar yang diinginkan
  \caption{Pose MediaPipe}
  \label{fig:Pose MediaPipe}
\end{figure}

\subsection*{Precision}
Precision merupakan rasio dimana TP (true positive) dimana jumlah positif benar dan FP (false positive) jumlah positif palsu sebagai metrik evaluasi dalam konteks machine learning, memberikan ukuran terhadap rasio prediksi positif yang tepat dibandingkan dengan seluruh prediksi positif yang diberikan oleh model. Dengan kata lain, precision 19 memberikan wawasan seberapa akurat model dalam membuat prediksi positif. Lebih rinci, precision mencerminkan seberapa sering model berhasil mengklasifikasikan instance sebagai positif dengan benar dalam keseluruhan dataset. Nilai precision dapat mengindikasikan sejauh mana model mampu memberikan prediksi yang benar dalam konteks positif. Rentang nilai precision berada antara 0 dan 1, di mana nilai 1 menunjukkan bahwa semua prediksi positif model adalah benar, sementara nilai 0 menunjukkan bahwa tidak ada prediksi positif yang benar.

\begin{equation}
    \frac{TP}{TP+FP}
\end{equation}

Pada persamaan diatas menyajikan perbandingan antara prediksi positif yang tepat dengan total prediksi positif yang diberikan oleh model, memberikan pandangan yang lebih mendalam terkait kemampuan model dalam menghasilkan hasil yang benar dalam kategori yang diinginkan

\subsection*{Recall}
Recall merupakan metrik yang digunakan untuk mengukur rasio dari data positif yangbenar yang ditemukan dari seluruh data positif. Recall memberikan informasi tentang seberapa baik model machine learning menemukan semua data positif. Nilai recall berkisar antara 0 dan 1. Recall yang tinggi menunjukan bahwa kelas yang dikenali dengan benar banyak, atau false negative yang didapatkan sedikit. Rumus dari recall dapat dilihat pada persamaan dibawah

\begin{equation}
    \frac{TP}{TP+FN}
\end{equation}

Recall merupakan rasio dimana TP (true positif ) adalah jumlah positif benar dan FN (false negative) jumlah negatif palsu

\subsection*{Mean Average Precision (mAP}
Mean Average Precision (mAP) adalah sebuah metrik akurasi yang dihasilkan dari menghitung rata-rata dari Average Precision (AP) atau presisi rata-rata. AP sendiri diperoleh melalui perhitungan precision dan recall. Oleh karena itu, mAP dapat dianggap sebagai metrik evaluasi yang sangat informatif dalam mengevaluasi kinerja suatu sistem.

\begin{equation}
    AP=\sum ((Recall_{n+1}-Recall_{n})\times Precision_{interp})
\end{equation}

\begin{equation}
  \times Recall_{n+1})
\end{equation}
\begin{equation}
    mAP=\frac{1}{n}\sum_{n}^{i=1}AP_{i}
\end{equation}

\subsection{Intersection over Union (IoU)}
\label{subsec:hukumnewton}

Intersection over Union, atau IoU, adalah metrik yang digunakan untuk mengevaluasi keakuratan posisi objek yang dideteksi oleh model dalam pemrosesan gambar. Prinsipnya adalah dengan menghitung area persinggungan antara kotak deteksi yang dihasilkan oleh model dengan kotak referensi yang merupakan standar emas atau Ground Truth. Rasio ini didapat dengan membandingkan area irisan kedua kotak tersebut terhadap keseluruhan area yang mereka cakup secara bersamaan. Jika kita membayangkan kedua kotak tersebut sebagai satu kesatuan, maka IoU memberikan kita sebuah skor yang mengukur seberapa baik model kita dalam memprediksi lokasi objek sebenarnya. Semakin besar area persinggungan relatif terhadap total area gabungan, semakin tinggi nilai IoU, yang menandakan keakuratan prediksi yang lebih baik. Secara Sistematis, hal ini dituliskan sebagai :

% Contoh pembuatan persamaan
\begin{equation}
IntersectionoverUnion(IoU) = \frac{\left |A\bigcap B  \right |}{\left | A\bigcup B \right |}.
\end{equation}

\subsection{Intel NUC}
Intel NUC (Next Unit of Computing) adalah solusi komputasi yang kompak dan kuat yang dirancang oleh Intel untuk memenuhi berbagai kebutuhan komputasi, mulai dari hiburan rumah hingga gaming dan tugas profesional. Intel NUC dengan fitur prosesor Intel Core generasi dalam form factor kompak 4x4 inci. Dirancang untuk menawarkan kombinasi ukuran, kinerja, keberlanjutan, dan keandalan yang dibutuhkan oleh bisnis modern. Intel NCU model tertentu juga menyertakan teknologi Intel vPro® Enterprise dengan keamanan yang ditingkatkan. Mini PC ini dapat diupgrade dan diperbaiki, menjadikannya pilihan serbaguna untuk berbagai aplikasi bisnis termasuk komputasi klien, komputasi edge, dan digital singage.


\subsection{ESP32 Devkit V1}

ESP32 Devkit V1 adalah salah satu development board yang dibuat oleh DOIT untuk menjalankan modul ESP-WROOM-32 buatan Espressif. ESP32 Devkit dikenal dengan Development board yang kaya fitur dengan konektivitas Wi-Fi dan Bluetooth terintegrasi untuk beragam aplikasi. Devkit ini memiliki banyak pin yang memungkinkannya untuk diprogram dengan banyak tugas.

\subsection{Motor Driver H-Bridge}

Driver motor H-Bridge adalah rangkaian elektronik yang digunakan untuk mengontrol arah dan kecepatan motor DC. Cara kerjanya berdasarkan empat switch yang membentuk jembatan H (H-Bridge), yang mana dengan mengatur pembukaan dan penutupan switch-switch ini, kita dapat mengatur arah arus yang mengalir ke motor. Dengan demikian, kita bisa mengubah arah putaran motor DC. Driver Motor H-Bridge tersusun oleh sekumpulan transistor yang berfungsi sebagai pengendali motor, terutama yang memerlukan arus serta tegangan yang cukup besar. selain itu, Rangkaian H-Bridge juga dapat memberikan fungsi pengereman pada motor dengan menghubungkan kedua terminal motor sehingga motor dapat berhenti lebih cepat. \cite{fibrianianalisis}

\subsection{Kursi Roda Elektrik KY-123}
Kursi roda Elektrik KY-123 merupakan alat bantu mobilitas yang terdiri dari struktur dasar kursi roda, sistem pengendalian gerak, mesin elektrik, dan modul baterai. Keunggulan alat ini terletak pada kemampuannya untuk dikendalikan dengan mudah dan nyaman, meminimalkan usaha fisik yang diperlukan pengguna dibandingkan dengan kursi roda manual. Ini sangat bermanfaat bagi individu dengan kondisi hemiplegia, memungkinkan pengoperasian dengan satu tangan. Selain itu, kursi roda elektrik ini juga memberikan solusi mobilitas yang lebih baik bagi lansia yang mengalami keterbatasan dalam bergerak.

\begin{figure}[H]
  \centering

  % Ubah dengan nama file gambar dan ukuran yang akan digunakan
  \includegraphics[scale=0.045]{gambar/kursi roda.png}

  % Ubah dengan keterangan gambar yang diinginkan
  \caption{Gambar Kursi Roda Elektrik KY-123}
  \label{fig:roketluarangkasa}
\end{figure}