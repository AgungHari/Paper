% Ubah judul dan label berikut sesuai dengan yang diinginkan.
\section{Pendahuluan}
\label{sec:pendahuluan}

% Ubah paragraf-paragraf pada bagian ini sesuai dengan yang diinginkan.
Mobilitas merupakan aspek penting dalam kehidupan sehari-hari yang memungkinkan individu untuk melakukan berbagai aktivitas dengan mandiri. Namun, bagi individu yang mengalami kelumpuhan atau gangguan mobilitas lainnya, ketergantungan pada alat bantu seperti kursi roda menjadi hal yang tidak terhindarkan. Menurut Kamus Besar Bahasa Indonesia (KBBI), lumpuh adalah kondisi kehilangan fungsi gerak pada bagian tubuh tertentu yang dapat disebabkan oleh berbagai faktor, seperti cedera atau penyakit saraf \cite{Daring_2016}. Dalam konteks medis, kelumpuhan dapat terjadi akibat kerusakan pada sistem saraf, baik itu saraf pusat maupun saraf tepi, yang mengakibatkan hilangnya kontrol motorik pada anggota tubuh \cite{Pansawira_2022}.

Perkembangan teknologi dalam bidang ini telah memberikan kontribusi signifikan dalam meningkatkan kualitas hidup penyandang disabilitas. Salah satu inovasi yang menonjol adalah pengembangan kursi roda elektrik yang dikendalikan melalui joystick. Penelitian oleh Choi, Chung, dan Oh menunjukkan bahwa kontrol gerak kursi roda elektrik yang diintegrasikan dengan joystick dapat meningkatkan keselamatan dan kenyamanan pengguna \cite{choi2019motion}. Meskipun demikian, tantangan tetap ada dalam hal navigasi dan penghindaran rintangan, terutama di lingkungan yang dinamis dan kompleks.

Seiring dengan kemajuan teknologi kecerdasan buatan, deep learning telah menjadi alat yang sangat efektif dalam mendeteksi, mengidentifikasi, dan melacak objek secara real-time. Lecrosnier dalam penelitiannya mengungkapkan bahwa penggunaan deep learning dalam deteksi dan pelokalisasian objek pada kursi roda pintar dapat meningkatkan mobilitas dan keselamatan pengguna di lingkungan kesehatan \cite{lecrosnier2021deep}. Teknologi ini memungkinkan kursi roda untuk secara otomatis menghindari rintangan, termasuk manusia, sehingga pengguna dapat bergerak dengan lebih aman dan efisien.

Pengembangan kursi roda otonom menjadi langkah vital untuk meningkatkan kemandirian dan kualitas hidup individu dengan keterbatasan mobilitas. Dengan perkembangan teknologi sensor dan pemrosesan gambar, aplikasi metode deteksi objek seperti YOLO (You Only Look Once) menjadi inti dari inovasi solusi mobilitas otonom.

Dengan latar belakang ini, proyek ini bertujuan untuk meningkatkan navigasi kursi roda otonom melalui integrasi YOLOV8 dalam deteksi dan penghindaran rintangan, serta memastikan keselamatan dan kenyamanan pengguna dalam berbagai situasi.
